\documentclass[a4paper]{kulakarticle}

\usepackage[utf8]{inputenc}
\usepackage[dutch]{babel}
\usepackage{titling}
\title{Kost van een Airbnb-verblijf}
\author{Project statistiek}
\date{Academiejaar 2022 -- 2023}
\address{
	Louis Vandenbruwaene\\
	Jasper Denorme\\
	Dieter Demunck}
\usepackage{graphicx,flafter,framed,caption}
\usepackage{amsfonts,amssymb,amsmath,textcomp,eurosym,wasysym}
\usepackage{listings}
\usepackage{siunitx}
\sisetup{output-decimal-marker={,}}
\usepackage{enumitem}
\usepackage{multicol}
\usepackage{caption}
\usepackage{tabularx,array}
\usepackage{times}
\usepackage{textcomp}
\newcommand{\rood}[1]{\textcolor{red}{#1}}

\begin{document}
	\maketitle
	\section*{Inleiding}
	Airbnb is een online platform (\href{https://www.airbnb.com}{\texttt{www.airbnb.com}}) waarop reizigers een kort verblijf kunnen boeken in een accommodatie (bv. een kamer, een huis, een woonboot, …) die door particulieren wordt verhuurd. Airbnb werd opgericht in 2008 en is ondertussen een erg populair alternatief geworden voor de traditionele hotels. \\
	
	Dit onderzoek focust op de totale kostprijs voor het huren van een Airbnb-verblijf in Amsterdam voor twee personen gedurende een weekend (vrijdag tot zondag). Er wordt nagegaan met welke factoren deze kostprijs samenhangt en in welke mate de kost daarmee kan worden voorspeld. Hierbij wordt er gebruik gemaakt van een aantal veranderlijken verzameld in het kader van een onderzoeksproject uitgevoerd door Kristóf Gyódi en Łukasz Nawaro. Het gaat enerzijds over kenmerken van het verblijf en tevredenheid van eerdere gasten volgens de gegevens van Airbnb, anderzijds over de ligging van het verblijf ten opzichte van het stadscentrum, bezienswaardigheden en restaurants, telkens rekening houdend met de populariteit, zoals gerapporteerd door TripAdvisor.\\
	
	Tabel \ref{beschrijving} beschrijft de variabelen en tabel \ref{uitreksel} geeft enkele basisstatistieken weer van de veranderlijken die gebruikt worden in het onderzoek. \\\\

\begin{table}[h]
	\centering
	\begin{tabular}{c|p{10cm}}
		\raggedright
		Naam & Beschrijving\\
		\hline
		realSum & Som van alle kosten in euro\\ 
		room & Soort verblijf (1 = volledige woning, 2 = afzonderlijke kamer, 
		3 = gedeelde kamer) \\ 
		capacity & Maximaal aantal gasten \\
		bedrooms & Aantal beschikbare slaapkamers in het verblijf \\
		dist & Afstand tot het stadscentrum (km) \\
		metro & Afstand tot dichtstbijzijnde metro-halte (km)\\
		attr & Attractiescore, nabijheid van bezienswaardigheden \\
		rest & Restaurantscore, nabijheid van restaurants \\ 
		host & Type verhuurder (0 = enige beschikbare woning, 1 = 2 tot 4 beschikbare woningen,
		2 = meer dan 4 beschikbare woningen) \\ 
		cleanliness & Modale score voor netheid van het verblijf volgens gasten (op 10) \\
		satisfaction & Tevredenheid van de gasten (op 10)\\
		
	\end{tabular}
	\caption{Beschrijving van de gebruikte veranderlijken.}
	\label{beschrijving}
\end{table}
\begin{table}[h]
	\centering
	\begin{tabular}{| l| l| l|  p{5cm} |}
		\hline
		Naam & Gemiddelde $\pm$ standaardfout  & Bereik & Algemene vorm\\  [1ex]
		\hline\hline
		realSum & $604.8\pm 443.6828 $ & $7964.755$  &  zeer rechtsscheef\\    [0.5ex]
		\hline
		capacity & $2.77\pm 1.019876$ & $4$  & rechtsscheef \\[0.5ex]
		\hline
		bedrooms & $1.303\pm 0.7329492$ & $5$ & rechtsscheef \\[0.5ex]
		\hline
		dist & $2.80634\pm 2.036602$  & $11.18089$ & zeer rechtsscheef \\  [0.5ex]
		\hline
		metro & $1.0892\pm 0.8265551$ & $4.375388$ & zeer rechtsscheef \\ [0.5ex]
		\hline
		attr & $2.101\pm 0.9407762$ & $9$  & rechtsscheef \\  [0.5ex]
		\hline
		rest & $3.285\pm 1.760009$ & $9$ & benaderend lognormaal met zware staarten \\ [0.5ex]
		\hline
		cleanliness & $9.471\pm 0.8306492$ & $8$ & benaderend exponentieel  \\ [0.5ex]
		\hline
	\end{tabular}
	\caption{Basisstatistieken van de gebruikte veranderlijken.}
	\label{uitreksel}
\end{table}
	
	\section{Methode}
	
	\subsection{Kenmerken van de steekproef}
	Om te testen of de gemiddelde kost veranderd is tegenover 2019, maken we gebruik van een student t-test voor één gemiddelde. We mogen deze gebruiken, want de dataset voldoet aan de centrale limietstelling (n = 977). Om na te gaan of het aandeel particuliere verhuurders groter is dan het aantal professionele verhuurders, maken we gebruik van een $z$-test voor één proportie.\\
	Via een $\chi^2_{3}$-test gaan we na of het aantal beschikbare kamers Poisson verdeeld is. Dit deden we nadat we data bij elkaar hebben gegooid, om aan de Cochranregel te voldoen.
	
	\subsection{Gemiddelde kost}
	
	Hier zal onderzocht worden of de totale prijs van een verblijf voor twee personen afhankelijk is van bepaalde andere variabelen.
	Aangezien de variantie bij alle onderzochte variabelen nooit gekend is, werd bij alle drie de vragen eerst de normaliteit van de twee categorieën getest, aan de hand van de Shapiro-Wilk-test. Hierna wordt een Student t-test voor twee gemiddelden bij verschillende varianties uitgevoerd, dit ook voor alle drie de vragen. Deze laatste test mag uitgevoerd worden, omdat de centrale limietstelling (CLS) voldaan is.

	\subsection{Associatie met de verschillende veranderlijken}
	
	Verder wordt de associatie tussen de kost van een verblijf en de andere gegevens onderzocht. Aangezien \verb|realSum| niet normaal verdeeld is (zie figuur \ref{fig:qqplotvrealsum}) en er zelden samenvallende waarden voorkomen, wordt de associatie tussen \verb|realSum| en de continue variabelen getest aan de hand van de Spearman-correlatietest. Om de afhankelijkheid met de discrete en kwalitatieve variabelen te bepalen, wordt een $\chi ^2$-test gebruikt. Hiervoor wordt de variabele \verb|realSum| gediscretiseerd, door deze op te delen in zijn vier kwantielen. Ook de discrete en kwalitatieve variabele worden samengevoegd om aan de Cochranregel te voldoen. Enkel \verb|host| blijft onveranderd. De opdelingen zijn te vinden in tabel \ref{tab:grenzen}. \\\\
	
	\begin{figure}
		\centering
		\includegraphics[width=0.7\linewidth]{Figuren/qqplotvrealSum}
		\caption{Kwantielplot van \textit{realSum}.}
		\label{fig:qqplotvrealsum}
	\end{figure}

	\begin{table}[h]
		\centering
			\begin{tabular}{c|c}
			\centering
			naam& opdelende grenzen\\
			\hline
			\textbf{cleanliness} & $ 0 $ - $ 7.5 $ - $ 8.5$ - $ 9.5 $ - $ 10.5$ \\
			\textbf{capacity}& $-0.5$ - $2.5$ - $3.5$ - $4.5$ - $6.5$\\
			\textbf{bedrooms}& $-0.5$ - $0.5$ - $1.5$ - $2.5$ - $6$\\
			\textbf{room} & $0.5$ - $1.5$ - $3.5$\\
			\end{tabular}
			\caption{De grenzen, die de gegevens opdelen zodat de aan Cochran-regel voldaan is.}
			\label{tab:grenzen}
	\end{table}
	\subsection{Verklaren van de kost}
	
	We zullen hier trachten \textit{realSum} te verklaren aan de hand van de andere veranderlijken. Dit gebeurt met een lineair regressiemodel. Eerst wordt het model gemaakt enkel op basis van de variabele \textit{attr}. Er wordt ook bestudeerd of een model gebaseerd op de $log_{10}$-transformatie van deze veranderlijken, $log_{10}$(\textit{attr}) en $log_{10}$(\textit{realSum}), een beter resultaat geeft, aangezien deze variabelen allebei rechtsscheef verdeeld zijn: skewness \textit{attr} $=$ $2,454883$ ; skewness \textit{realSum} $=6.496029$.\\
	Daarna wordt er via achterwaartse regressie een meervoudig regressiemodel gezocht, waar we alle continue variabelen voor gebruiken en ook gebruik maken van  $log_{10}$-transformaties. Ten slotte wordt nog bestudeerd of de binaire variabele \textit{dummy}, \textit{room} == volledige woning, een significante invloed heeft op het model. Voor alle modellen wordt de kwaliteit ook onderzocht, dit doen we aan de hand van ingebouwde functies in R. 
	
	
	\section{Resultaten}
	
	\subsection{Kenmerken van de steekproef}
	De gemiddelde kost in de steekproef is 604 euro, terwijl deze in 2019 620 euro was. Het verschil van 16 euro is niet significant ($t_{976}$ = -1.1, p = 0.29). We kwamen uit dat $p \approx 0$ voor onze test op het aandeel van particuliere vs. professionele verhuurders. We bekomen een verschil van 295. Bij de $\chi$-kwadraat test voor onze verdeling bekomen we $\chi^2 = 424.77$ en $p \approx 0$, we kunnen dus aannemen dat het aantal beschikbare kamers niet Poisson verdeeld is.
	
	\subsection{Gemiddelde kost}
	
	We merken uit de gegevens in tabel \ref{tab:intermediary_results_gemiddelde_kost} dat elke categorie minstens 200 datapunten bevat, waardoor we kunnen veronderstellen dat de centrale limietstelling (CLS) geldig is. Uit de resultaten van de Shapiro-Wilk-testen volgt ook dat we met aan zekerheid grenzende waarschijnlijkheid kunnen veronderstellen dat geen enkele categorie normaal verdeeld is.
	
	Uit de resultaten van de t-testen bij verschillende varianties in tabel \ref{tab:end_results_gemiddelde_kost} merken we dat er op basis van de steekproef met aan zekerheid grenzende waarschijnlijkheid een verschil bestaat tussen de gemiddelde kost voor een verblijf van twee personen bij verblijven waarbij de volledige woning wordt verhuurd ten opzichte van verblijven waar maar een deel wordt verhuurd, waarbij volledige woningen gemiddeld €185 duurder zijn. We merken ook dat er een redelijke aanwijzing is dat het aantal verblijven dat een gastheer verhuurt een verschil zou betekenen in prijs. Hierbij zijn verblijven bij gastheren met meerdere woningen gemiddeld ongeveer €52 goedkoper.
	
	Over het verschil in prijs voor twee personen voor verblijven met of zonder de maximum netheidsscore valt weinig te concluderen, aangezien de p-waarde net onder de verwerpingsgrens ligt. Dit zou toeval kunnen zijn, maar zou ook een aanwijzing kunnen zijn voor een werkelijk verschil in prijs, waarbij verblijven onder de maximum netheidsscore gemiddeld ongeveer €35 goedkoper zouden zijn.
	
	\begin{table}
		\caption{Aantal datapunten, gemiddelde, en p-waarde van de Shapiro-Wilk-test (uitgevoerd op deze datapunten) van de kost voor een verblijf van 2 personen, gefilterd op bepaalde gegevens.}
		\label{tab:intermediary_results_gemiddelde_kost}
		\begin{tabular}{|l*{3}{|c}|}
			\hline
			Kost voor 2 personen bij ...        & Aantal observaties & Gemiddelde prijs & Shapiro-Wilk-test p-waarde \\ \hline
			\hline
			Maximumscore netheid                & 376    & $457.1111$ & $< 2.2 \cdot 10^{-16}$ \\ \hline
			Niet maximumscore netheid           & 215    & $422.2669$ & $3.091 \cdot 10^{-12}$ \\ \hline
			Gastheer verhuurt 1 woning          & 367    & $463.9553$ & $< 2.2 \cdot 10^{-16}$ \\ \hline
			Gastheer verhuurt meerdere woningen & 224    & $412.4534$ & $< 2.2 \cdot 10^{-16}$ \\ \hline
			Volledige woning wordt verhuurd     & 288    & $539.0370$ & $< 2.2 \cdot 10^{-16}$ \\ \hline
			Deel van de woning wordt verhuurd   & 303    & $354.5165$ & $< 2.2 \cdot 10^{-16}$ \\ \hline
		\end{tabular}
	\end{table}

	\begin{table}
		\caption{Verschil in gemiddelden, en p-waarde van de t-testen (bij verschillende varianties) op het verschil tussen twee categorieën van de kost van een verblijf voor twee personen.}
		\label{tab:end_results_gemiddelde_kost}
		\begin{tabular}{|l*{2}{|c}|}
			\hline
			Verschil in kost voor 2 personen tussen ...  & Verschil van gemiddelden
			& t-test p-waarde \\ \hline
			\hline
			
			Maximumscore voor netheid of niet                 & $34.84423$
			& $0.03848$              \\ \hline
			Gastheer verhuurt 1 woning of meerdere       & $51.50189$
			& $0.002512$             \\ \hline
			Gastheer verhuurt volledige woning of deel   & $184.5205$
			& $< 2.2 \cdot 10^{-16}$ \\ \hline
		\end{tabular}
	\end{table}
	
	\subsection{Associatie met de verschillende veranderlijken}
	
	Uit tabel \ref{continue variabelen afhankelijkheid} kunnen we met aan zekerheid grenzende waarschijnlijkheid concluderen dat \textit{realsum} afhankelijk is van elke continue variabele in de dataset. De bijhorende correlatie wordt geschat door rho.
	\begin{table}[h]
		\centering
		\begin{tabular}{c|c|c|c }
			naam & p-waarde & afhankelijkheid & rho\\
			\hline
			\hline
			attr &$ < 2.2\cdot 10^{-16}$& ja&$0.4314979 $ \\
			rest &$ < 2.2\cdot 10^{-16}$& ja&$0.4276987 $ \\
			satisfaction &$ 1.657\cdot 10^{-7}$& ja&$0.1664983 $ \\
			metro &$ 9.476\cdot 10^{-10}$& ja& $-0.1941159$ \\ 
			dist & $< 2.2 \cdot 10^{-16} $&ja& $-0.4107378$ \\
		\end{tabular}
		\caption{Afhankelijkheid van de prijs met de continue variabelen.}
		\label{continue variabelen afhankelijkheid}
	\end{table}
	Uit tabel \ref{discrete variabelen afhankelijkheid} kunnen we met aan zekerheid grenzende waarschijnlijkheid besluiten dat \textit{realsum} afhankelijk is van: \textit{capacity}, \textit{bedrooms}, \textit{room} en \textit{host}. We kunnen ook met grote waarschijnlijkheid zeggen dat \textit{cleanliness} niet gecorreleerd is met \textit{realsum}. 
	
	\begin{table}[h]
		\centering
		\begin{tabular}{c|c|c|c }
			naam & p-waarde & afhankelijkheid & $\chi ^2$\\
			\hline
			\hline
			capacity &$ < 2.2\cdot 10^{-16}$& ja& $390.3$ \\ 
			bedrooms &$ < 2.2\cdot 10^{-16}$& ja&$368.1 $ \\
			room &$ < 2.2\cdot 10^{-16}$& ja&$ 337.29$ \\
			host &$ 5.506\cdot 10^{-7}$& ja&$39.581 $ \\
			cleanliness & $0.7775$&nee& $5.6174$ \\
		\end{tabular}
		\caption{Afhankelijkheid van de prijs met de discrete en kwalitatieve variabelen.}
		\label{discrete variabelen afhankelijkheid}
	\end{table}
	
	\subsection{Verklaren van de kost}
	
	Bij het uitvoeren van het eerste regressiemodel, waarbij enkel \textit{attr} in rekening gehouden wordt, bekomen we: 
    $ y = 338.93 + 126.58\cdot x$. Hier is $y$ de variabele \textit{realSum} en $x$ de variabele \textit{attr}. Laten we dit model de naam 'simpelmodel1' geven. Vervolgens wordt \textit{attr} vervangen door $log_{10}$(\textit{attr}) en \textit{realSum} door $log_{10}$(\textit{realSum}). Dit model noemen we 'simpelmodel2'. De bekomen vergelijking is dan: $ y = 2.55138 + 0.56912\cdot x$. Hier is $y = log_{10}$(\textit{realSum}) en $x = log_{10}$(\textit{attr}).\\
	
	In het naïeve model gebruiken we alle continue veranderlijken om te prijs te voorspellen. Maar zoals eerder vermeld zal hier achterwaartse regressie op toegepast worden. De variabelen \textit{rest} en \textit{metro} zullen dus verwijderd worden, omdat deze niet met significante zekerheid \textit{realSum} voorspellen. De p-waarde van \textit{rest} is namelijk $0.759213$ onder de nulhypothese dat het onafhankelijk is met \textit{realSum}. De p-waarde van \textit{metro} onder diezelfde nulhypothese is $0.455561$. Bij deze tweede waarde is \textit{rest} al uit het model verwijderd. Uiteindelijk zijn de andere drie veranderlijken wel significant voorspellend. Het bekomen model ziet er dus als volgt uit: $y = -190.900 + 75.886\cdot x_1 + 79.480\cdot x_2 -31.985\cdot x_3$. Hier is $y =$\textit{realsum}, $x_1 = $ \textit{satisfaction}, $x_2 =$ \textit{attr} en $x_3 =$ \textit{dist}. De bijhorende adjusted $R^2$ is zoals te zien in tabel \ref{rsq} tamelijk laag. De residuen zijn gemiddeld ongeveer 0, het grote verschil, te zien op figuur \ref{fig:residuplotsnaief}, in de uiteinden komt omdat er weinig meetwaarden zijn. We zien tevens dat de grootteorde van de afwijking ook niet constant is, zelfs in de grote puntenwolk zien we een stijgende lijn. Kijken we naar het Q-Q-plot van de residuen (\ref{fig:residuplotsnaief}), dan zien we al sterk afwijkende staarten en de fit lijkt systematisch te onderschatten, te overschatten en dan weer te onderschatten. De residuen lijken dus niet normaal verdeeld te zijn. \\
\begin{figure}[h]
	\centering
	\includegraphics[width=0.8\linewidth]{Figuren/residuplotsnaief}
	\caption{Residuplots voor het naief model, ze wijzen sterk op een niet normale verdeling.}
	\label{fig:residuplotsnaief}
\end{figure}
	
	In logattrmodel proberen we eens \textit{attr} te vervangen door zijn loggetransformeerde. Hier wordt enkel \textit{metro} verwijderd. De adjusted $R^2$ is helaas niet gestegen ten opzichte van naïefmodel. Dit is dus geen beter model. Onze zoektocht naar een optimaal voorspellend model gaat verder. \\
	
	Vanaf hier zullen alle modellen \textit{realSum} transformeren tot $log_{10}$(\textit{realSum}). Dit zal een merkbaar verschil maken in de voorspellende kracht van de modellen. Ten eerste worden alle veranderlijken gebruikt. \textit{rest} zal meteen niet significant blijken in logmodel. \textit{metro} is een randgeval ( p = $0.0449$) en zal dus eens verwijderd worden in logmodel2. Dit verlaagt de adjusted $R^2$ en de kwaliteit van het model. \\
	
	Tot slot combineren we alles tot onsmodel. Hier nemen we zowel van \textit{realSum} als van \textit{attr} de logtransformatie. \textit{Rest} en \textit{metro} zijn opnieuw niet significant en zullen niet voorkomen in onsmodel. De residuen van het bekomen model gedragen zich ook al beter, ook hier lijken ze gemiddeld 0 te zijn met een bijna volledig constante afwijking. Als we naar de Q-Q-plot kijken van de residuen, zien we dat er nog wel redelijk sterk afwijkende staarten zijn, maar de residuen lijken nu wel de normale verdeling beter te volgen.  
\begin{figure}[h]
	\centering
	\includegraphics[width=0.7\linewidth]{Figuren/residuplotsonsmodel}
	\caption{De residuplots voor 'onsmodel', de residuen volgen de normale verdeling al meer. Over het algemeen kunnen we wel nog niet besluiten dat deze normaal verdeeld zijn.}
	\label{fig:residuplotsonsmodel}
\end{figure}

	\begin{table}[h]
		\centering
		\begin{tabular}{c|c|c}
		\centering
		naam van het model & adjusted $R^2$ & voorschrift van het lineair model \\
		\hline
		 simpelmodel1 & $0.07109$ & \textit{realSum} $\sim$ \textit{attr}\\
		 simpelmodel2 &$0.1752$ & $log_{10}$(\textit{realSum}) $\sim$ $log_{10}$(\textit{attr}) \\
		naïefmodel & $0.09559$& \textit{realSum} $\sim$ \textit{attr}$+$\textit{satisfaction}$+$\textit{dist}\\
		 logattrmodel &$0.09559$ & \textit{realSum} $\sim$ $log_{10}$(\textit{attr})$+$\textit{satisfaction}$+$\textit{rest}$+$\textit{dist}\\
		 logmodel &$0.1936$ & $log_{10}$(\textit{realSum}) $\sim$ \textit{attr}$+$\textit{satisfaction}$+$\textit{metro}$+$\textit{dist}\\
		 logmodel2 &$0.1911$ & $log_{10}$(\textit{realSum}) $\sim$ \textit{attr}$+$\textit{satisfaction}$+$\textit{dist}\\
		  onsmodel &$0.2006$ & $log_{10}$(\textit{realSum}) $\sim$ $log_{10}$(\textit{attr})$+$\textit{satisfaction}$+$\textit{dist}\\
		\end{tabular}
		\caption{Alle gebruikte modellen met hun voorschrift.}
		\label{rsq}
	\end{table}
	

	
	\section{Discussie}
	
	\subsection{Kenmerken van de steekproef}
	Op basis van de steekproef lijkt er geen verandering te zijn in de kostprijs van een kamer. Met aan zekerheid grenzende waarschijnlijkheid kunnen we aannemen dat er meer particulier verhuurders zijn, het lijkt er dus op dat er een 'strenge' regulering is in Amsterdam. We kunnen ook met zeer grote waarschijnlijkheid aannemen dat het aantal beschikbare kamers niet Poisson verdeeld is.
	
	\subsection{Gemiddelde kost}
	
	% I feel like this is maybe too much....
	Uit de gegevens merken we dat de prijs van een verblijf voor twee personen praktisch zeker duurder wordt wanneer de volledige woning wordt verhuurd, wat op zich intuïtief is.
	Er is ook een duidelijke aanwijzing dat verblijven met een gastheer die meerdere woningen verhuurd goedkoper zou zijn. Een mogelijke verklaring is dat deze gastheren voorraden aankopen voor meerdere verblijven, en dus mogelijks goedkoper grotere aantallen voorraden kunnen aankopen en deze dan kunnen verdelen onder hun verblijven. Een andere verklaring is dat zij mogelijks meer competitief zijn in hun prijs, aangezien dit hun hoofdinkomen zou zijn. In tegenstelling tot een gastheer die een verblijf verhuurt als bijzaak, moeten gastheren die dit professioneel doen een zekerheid hebben op huurders.
	Er is een grote onzekerheid of er een werkelijk verschil is in prijs tussen verblijven met en zonder de maximumscore voor netheid. Een mogelijke verklaring is dat gastheren zonder de maximumscore zelf niet goed inzien hoe vuil hun verblijf is in vergelijking met propere verblijven. Aangezien vele verblijven een netheidsscore tussen 8 en 10 hebben (zie figuur \ref{fig:cleanliness}), worden meeste gasten hier misschien weinig door afgeschrokken. Een verschil zou dan te verklaren kunnen zijn doordat vuile verblijven hun prijs moeten verlagen om een redelijk aantal gasten te kunnen hebben.
	
	\begin{figure}
		\centering
		\includegraphics{Figuren/cleanliness_hist.pdf}
		\caption{Histogram van de netheidsscore}
		\label{fig:cleanliness}
	\end{figure}
	
	\subsection{Associatie met de verschillende veranderlijken}
	
	Op basis van de steekproef zijn \textit{attr}, \textit{rest}, \textit{capacity}, \textit{room}, \textit{host}, \textit{bedrooms} en \textit{satisfaction} positief gecorreleerd met \textit{realsum}. Bij de continue variabelen kunnen we dit afleiden uit de rho en bij de andere kunnen enkel kijken naar de scatterplots. Daarentegen zijn \textit{dist} en \textit{metro} negatief gecorreleerd met de totale prijs. Dit is duidelijk uit de waarde van rho. Intuïtief is dit ook aanvaardbaar. Toegankelijkheid tot het stadscentrum en een metrohalte verhoogt de waarde van een verblijf.

	\begin{figure}
		\centering
		\includegraphics[width=0.49\linewidth]{Figuren/correlatievanattr}
		\includegraphics[width=0.49\linewidth]{Figuren/correlatievanrest}
		\includegraphics[width=0.49\linewidth]{Figuren/correlatievanbedrooms}
		\includegraphics[width=0.49\linewidth]{Figuren/correlatievancapacity}
		
		\caption{Scatterplots voor \textit{realSum} in functie van de meest significante veranderlijken.}
		\label{fig:correlatievanattr}
	\end{figure}




	\subsection{Verklaren van de kost}
	
	In onze zoektocht naar het optimale model, die \textit{realSum} kan voorspellen aan de hand van de andere veranderlijken in de dataset, zijn we terechtgekomen op onsmodel. Met een adjusted $R^2$ van $20.06$ procent geeft dit een tamelijk slechte voorspelling van de totale prijs, maar het is al veel beter dan alle vorige modellen. De verbetering is te zien op figuur \ref{fig: }.%hier moet nog een figuur
	\\
	Het uiteindelijke model heeft de volgende vorm:
	\begin{equation}
		log_{10}(\textit{realSum}) = 2.209482 + 0.046029\cdot \textit{satisfaction} + 0.402077\cdot log_{10}(\textit{attr}) -0.016323\cdot \textit{dist}
	\end{equation} 

 
	\begin{figure}
		\centering
		\includegraphics[width=0.7\linewidth]{Figuren/naiefmodel}
		\caption{Het naïeve model}
		\label{fig:naiefmodel}
	\end{figure}
	
	\begin{figure}
		\centering
		\includegraphics[width=0.7\linewidth]{Figuren/onsmodel}
		\caption{Het uiteindelijke model}
		\label{fig:onsmodel}
	\end{figure}
	
	\begin{figure}
		\centering
		\includegraphics[width=0.7\linewidth]{Figuren/betr.attr}
		\caption{Het onderscheid tussen \textit{attr} en $log_{10}$(\textit{attr}).}
		\label{fig:betr}
	\end{figure}


	
	
	
	\section*{Besluit}
	
	
\end{document} 